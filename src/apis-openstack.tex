\documentclass{beamer}
\usetheme[pageofpages=de,% String used between the current page and the
                         % total page count.
          bullet=circle,% Use circles instead of squares for bullets.
          titleline=true,% Show a line below the frame title.
          alternativetitlepage=true,% Use the fancy title page.
          titlepagelogo=../img/fse_ministerio_ancho_texto.png%,% Logo for the first page.
	  %watermark=junta_girado,
          %watermarkheight=75px,% Height of the watermark.
          %watermarkheightmult=4,% The watermark image is 4 times bigger
                                % than watermarkheight.
          ]{Torino}

\usepackage[spanish]{babel} % Para separar correctamente las palabras
\usepackage[utf8]{inputenc} % Este paquete permite poner acentos y eñes usando codificación utf-8

\usepackage{color}
\usepackage{listings}
\definecolor{javared}{rgb}{0.6,0,0} % for strings
\definecolor{javagreen}{rgb}{0.25,0.5,0.35} % comments
\definecolor{javapurple}{rgb}{0.5,0,0.35} % keywords
\definecolor{javadocblue}{rgb}{0.25,0.35,0.75} % javadoc

\author{
\small{IES Gonzalo Nazareno}\\
\tiny{Dos Hermanas (Sevilla)}\\
\small{IES Los Albares}\\
\tiny{Cieza (Murcia)}\\
\small{IES La Campiña}\\
\tiny{Arahal (Sevilla)}\\
\small{IES Ingeniero de la Cierva}\\
\tiny{Murcia}\\
\vspace{.5cm}
\includegraphics[width=0.2\textwidth]{cc_by_sa.png}}
\title{Utilización de las APIs de OpenStack}
\institute{Proyecto de Innovación\\ {\color{white} .\\} \emph{Implantación y puesta a punto de la infraestructura de un cloud computing privado para el despliegue de servicios en la nube}} 

\lstset{basicstyle=\small,
aboveskip=5pt,
basicstyle=\tiny\ttfamily,
belowskip=5pt
}

\lstset{language=python,
basicstyle=\tiny\ttfamily,
keywordstyle=\color{javapurple}\bfseries,
stringstyle=\color{javared},
commentstyle=\color{javagreen},
morecomment=[s][\color{javadocblue}]{/**}{*/},
%numbers=left,
%numberstyle=\tiny\color{black},
%stepnumber=2,
%numbersep=10pt,
tabsize=4,
showspaces=false,
showstringspaces=false}

\begin{document}
\begin{frame}[t,plain]
\titlepage
\end{frame}

\begin{frame}
   \frametitle{REST (RESTful web API)}
  \begin{itemize}
  \item \emph{Representational State Transfer} (REST)
  \item Utilizado masivamente en Internet para la transferencia automática y
    controlada de información
  \item Utiliza HTTP para la comunicación entre el cliente y el servidor
  \item Se define una URI base en el servidor
  \item Comunicación entre cliente y servidor:
    \begin{itemize}
    \item El cliente realiza una petición HTTP (GET, POST, PUT o DELETE)
    \item El servidor contesta con un mensaje en un determinado
      formato (los más usados son XML y JSON)
    \end{itemize}
  \item Es más sencillo de implementar que otro protocolos como SOAP y está
    utilizándose de forma muy amplia
  \item Google code, Yahoo Developer Network, twitter API, \ldots
  \end{itemize} 
\end{frame}

\begin{frame}
  \frametitle{APIs de OpenStack}
  \begin{itemize}
  \item Los diferentes servicios de OpenStack se comunican entre sí mediante
    APIs RESTful
  \item Para cada API se define una URL:
    \begin{description}
    \item[keystone] \texttt{http://192.168.222.1:5000/v2.0}
    \item[glance] \texttt{http://192.168.222.1:9292/v1}
    \item[Compute] \texttt{http://192.168.222.1:8774/v2}
    \item[Volume] \texttt{http://192.168.222.1:8776/v1}
    \item[EC2] \texttt{http://192.168.222.1:8773/services/Cloud}
    \end{description}
  \item Algunos servicios distinguen entre URL para administrar, red privada y
    red pública
  \item Estas URLs se especifican durante la configuración de keystone
  \item La descripción de los métodos de cada API están disponibles en \url{http://api.openstack.org}
  \end{itemize}
\end{frame}

\begin{frame}[fragile]
  \frametitle{Ejemplo de utilización}
  \begin{itemize}
  \item Los propios clientes de OpenStack de línea de comandos utilizan las
    respectivas APIS:
  \end{itemize}
    \begin{center}
  \begin{lstlisting}
$ nova --debug list
connect: (172.22.222.1, 5000)
send: 'POST /v2.0/tokens HTTP/1.1\r\nHost: 172.22.222.1:5000\r\nContent-Length:124
\r\ncontent-type: application/json\r\naccept-encoding: gzip, deflate\r\naccept: ap
plication/json\r\nuser-agent: python-novaclient\r\n\r\n{"auth": {"tenantName": "te
st", "passwordCredentials": {"username": "user", "password": "testpass"}}}'
reply: 'HTTP/1.1 200 OK\r\n'
connect: (172.22.222.1, 8774)
send: u'GET /v2/aaaaaaaa5894473c8a98f89a895c6b2c/servers/detail HTTP/1.1\r\nHost: 
172.22.222.1:8774\r\nx-auth-project-id: test\r\nx-auth-token: e9233fef4ce34ee49f7d
b1aaaaaaa13f\r\naccept-encoding: gzip, deflate\r\naccept: application/json\r\nuser
-agent: python-novaclient\r\n\r\n'
reply: 'HTTP/1.1 200 OK\r\n'
+--------------------------------------+--------+---------------+----------------+
|                  ID                  |  Name  |     Status    |    Networks    |
+--------------------------------------+--------+---------------+----------------+
| b1724bd0-34f4-4bf1-9444-110eb3531602 | demo9  | VERIFY_RESIZE | vlan5=10.0.5.6 |
| e82814aa-fb1d-4c29-81ab-c39f99184413 | demo10 | ACTIVE        | vlan5=10.0.5.3 |
+--------------------------------------+--------+---------------+----------------+
  \end{lstlisting}
\end{center}
\end{frame}

\begin{frame}[fragile]
  \frametitle{Ejemplo de aplicación propia}
  \begin{lstlisting}[language=python]
#!/usr/bin/python
# -*- coding: utf-8 -*-

import requests
import json
from getpass import getpass
import ConfigParser

def obtener_token(url,user,passwd):
    """
    Recibe usuario y password de Keystone y devuelve el token de sesion
    """
    cabecera1 = {'Content-type': 'application/json'}
    datos = '{"auth":{"passwordCredentials":{"username": "%s", "password": \
            "%s"}, "tenantName":"service"}}' % (user,passwd)
    solicitud = requests.post(url+'tokens', headers = cabecera1, data=datos)
    if solicitud.status_code == 200:
        token = json.loads(solicitud.text)["access"]["token"]["id"]
        return token

url = config.get("keystone","url")
while True:
    adminuser = raw_input("Usuario de Keystone: ")
    adminpass = getpass("Password: ")
    admintoken = obtener_token(url,adminuser,adminpass)
    if len(admintoken) != 0:
        break
  \end{lstlisting}
\end{frame}

\begin{frame}
  \frametitle{El ecosistema}
  \begin{itemize}
  \item La utilización de APIs propicia la creación de muchas aplicaciones no
    oficiales
  \item Es muy fácil comunicarse con cualquier componente de OpenStack, por lo
    que cualquiera puede implementar una funcionalidad nueva, automatizar
    procesos o hacer aplicaciones completas.
  \item Ejemplos: ceilometer, moniker, heat, \ldots
  \item La filosofía abierta y libre de OpenStack hace que algunas de estas
    aplicaciones se incluyan posteriormente en la versión oficial
    (quantum en Folsom o ceilometer en Grizzly)
  \item Estilo OpenStack:
    \begin{itemize}
    \item Python
    \item API
    \item Integración continua (jenkins)
    \item Licencia Apache
    \item Github
    \end{itemize}
  \end{itemize}
\end{frame}

\end{document}
